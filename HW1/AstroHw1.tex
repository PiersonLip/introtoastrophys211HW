\documentclass{article}
\usepackage{cancel}
\usepackage{gensymb}
\title{Astro HW 1}
\author{Pierson Lipschultz}

\begin{document}
\maketitle
\section{}
\subsection{}
\begin{itemize}
    \item A) Collaboration and working with fellow students is encouraged, however, work still must be that of the individual, and copying answers is not allowed. 
    \item B) Copying from others, cheating, enabling others to cheat, all are violations of academic honesty.
    \item C) Because they are the words/work of someone other than myself, meaning that I am not truly learning the material. 
\end{itemize}
\subsection{}

\[tan^{-1}(\frac{61}{384}) \approx 9.0\deg\] \\
\[9.0 * \frac{\pi}{180} \approx .16rad\]

\subsection{}

\[G = 6.67430 * 10^{-11} m^{3} kg^{-1} s^{-2} = 39.5 AU^{3} M_{\odot}^{-1} yr^{-2}\]

\[6.67430 * 10^{-11} m^{3} kg^{-1} s^{-2} * 1.989 * 10^{30}kg M_{\odot}\]

\[6.67430 * 10^{-11} m^{3} \cancel{kg^{-1}} s^{-2} * 1.989 * 10^{30}\cancel{kg} M_{\odot}\]

\[6.67430 * 10^{-11} m^{3} s^{-2} * 1.989 * 10^{30} M_{\odot} * {(1.496 * 10^{11})}^3 AU^3 m^{-3}\]

\[6.67430 * 10^{-11} \cancel{m^{3}} s^{-2} * 1.989 * 10^{30} M_{\odot} * {(1.496 * 10^{11})}^3 AU^3 \cancel{m^{-3}}\]

\[6.67430 * 10^{-11} s^{-2} * 1.989 * 10^{30} M_{\odot} * {(1.496 * 10^{11})}^3 AU^3 * {(3.154 * 10^7)}^2 s^{2} y^{-2}\] 

\[6.67430 * 10^{-11} \cancel{s^{-2}} * 1.989 * 10^{30} M_{\odot} * {(1.496 * 10^{11})}^3 AU^3 * {(3.154 * 10^7)}^2 \cancel{s^{2}} y^{-2}\] 

\[6.67430 * 10^{-11} * 1.989 * 10^{30} M_{\odot} * {(1.496 * 10^{11})}^3 AU^3 * {(3.154 * 10^7)}^2 y^{-2}\] 

\[ = 39.5 AU^{3} M_{\odot}^{-1} yr^{-2}\]

\subsection{}

\subsubsection{a}

\[\frac{x}{\hat{s}} = \cos(\delta)\]
\[x = \cos(\delta)\]
\[x = \hat{x}\cos(\delta)\]

\[\frac{z}{\hat{s}} = \sin(\delta)\]
\[z = \sin(\delta)\]
\[z = \hat{z}\sin(\delta)\]

\[\hat{s} = \hat{x}\cos(\delta) + \hat{z}\sin(\delta) \]

\subsubsection{b}

\[\frac{z'}{\hat{x}} = \cos(\ell)\]
\[z' = \cos(\ell)\]
\[z' = \hat{z'}\cos(\ell)\]

\[\frac{x}{\hat{x}} = \sin(\ell)\]
\[x' = \sin(\ell)\]
\[x' = \hat{x'}\sin(\ell)\]

\[\hat{x} = \hat{z'}\cos(\ell) + \hat{x'}\sin(\ell)\]

\subsubsection{c}

\[\frac{z'}{\hat{z}} = \cos(90 - \ell)\]
\[z' = \cos(90 - \ell)\]
\[z' = \hat{z'}\cos(90 - \ell)\]

\[\frac{-x'}{\hat{z}} = \sin(90 - \ell)\]
\[-x' = \sin(90 - \ell)\]
\[-x' = \hat{x'}\sin(90 - \ell)\]

\[\hat{z} = \hat{z'}\cos(90 - \ell) - \hat{x'}\sin(90 - \ell)\]

\[\hat{z} = \hat{z'}\cos(90 - \ell) - \hat{x'}\sin(90 - \ell)\]

\[\hat{z} = \hat{z'}\sin(\ell) - \hat{x'}\cos(\ell)\]

\subsubsection{d}

\[\hat{s} = \hat{x}\cos(\delta) + \hat{z}\sin(\delta) \]

% \[\hat{s} = (\hat{z'}\cos(\ell) + \hat{x'}\sin(\ell))\cos(\delta) + (\hat{z'}\sin(\ell) - \hat{x'}\cos(\ell))\sin(\delta)\]

\[\hat{s} = \cos(\delta)(\hat{z'}\cos(\ell) + \hat{x'}\sin(\ell)) + \sin(\delta)(\hat{z'}\sin(\ell) - \hat{x'}\cos(\ell))\]

\[\hat{s} = \hat{z'}\cos(\ell)\cos(\delta) + \hat{x'}\sin(\ell)\cos(\delta) + \hat{z'}\sin(\ell)\sin(\delta) - \hat{x'}\cos(\ell)\sin(\delta)\]

\[\hat{s} = \hat{z'}(\cos(\ell)\cos(\delta) + \sin(\ell)\sin(\delta)) + \hat{x'}(\sin(\ell)\cos(\delta) - \cos(\ell)\sin(\delta))\]

\[\hat{s} = \hat{z'}\cos(\ell +\delta) + \hat{x'}\sin(\ell-\delta)\]

\subsubsection{e}

\[\theta = \frac{\pi}{2} - (\ell + \delta), \ell < \delta\]

\[\theta = \frac{\pi}{2} - (\ell - \delta), \ell > \delta\]

\subsubsection{f}

\[\theta_{NCP} = \ell\]
\subsubsection{g}

\[\delta_{lim} = 40.11\overline{6} \degree - 90\degree\]
\[\delta_{lim} \approx -49.884\degree \]
\[\delta_{lim} \approx -49\degree 53'\]

Yes, you can see  NGC 1316 from Champaign-Urbana. This is because $\delta_{NGC 1316} > \delta_{lim}$, or $37\degree12.5' > 49\degree 53'$, meaning that is it above the horizon at some point.

\subsubsection{h}
\[\delta = \ell\]

\subsection{}
\[d\Omega =\sin\theta d\theta d\phi\]
\[\int d\Omega = \int_{0}^{2\pi}\int_{0}^{\theta_{r}}\sin\theta d\theta d\phi\]
\[\Omega = \int_{0}^{2\pi}(-\cos\theta_{r} + 1) d\phi\]
\[\Omega = (-\cos\theta_{r} + 1)\]
\[\approx2\pi(-\cancel{1} -\frac{\theta^2}{2} +\cancel{1})\]
\[\Omega = -\pi\theta^2\]

\end{document}