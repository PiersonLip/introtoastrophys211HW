\documentclass{article}
\usepackage{cancel}
\usepackage{gensymb}
\usepackage{tikz}
\usepackage{amsmath}
\usetikzlibrary{angles,quotes}
\usetikzlibrary{arrows.meta,calc}

\renewcommand{\thesubsubsection}{\alph{subsubsection})}

\def\hwNUM{8}

\title{Astro HW \hwNUM}
\author{Pierson Lipschultz}

\sloppy
\begin{document}
\maketitle


\setcounter{section}{\hwNUM}
\subsection{Stellar Properties}

Consider two stars A and B with properties \dots

\begin{table}[htbp]
\makebox[\linewidth][c]{%
\begin{tabular}{lcccccc}
\hline
\textbf{Name} & \textbf{Spectral type} & $m_B$ (mag) & $m_V$ (mag)
& \textbf{Diameter} ($\times 10^{-3}$") & \textbf{Parallax} (") & \textbf{Mass} ($M_\odot$) \\
\hline
Star A & M2 Ia & 2.27 & 0.42 & 43.43 & 0.00451 & 19.6 \\
Star B & M2 V  & 8.96 & 7.52 & 1.43 & 0.393   & 0.46 \\
\hline
\end{tabular}}
\end{table}

\subsubsection*{(a) Distances}

\[
d(\mathrm{pc}) = \frac{1}{\theta_\mathrm{parallax}('')}
\]

\[
d_A = \frac{1}{0.00451} \approx 221.72~\mathrm{pc}, \qquad
d_B = \frac{1}{0.393} \approx 2.544~\mathrm{pc}
\]

\subsubsection*{(b) Radii}

From the small-angle relation
\[
\theta_\mathrm{diam} = \frac{2R}{d} \qquad \Rightarrow \qquad R = \frac{\theta_\mathrm{diam}}{2}\,d
\]
with $\theta$ in radians ($1'' = 4.848\times10^{-6}$~rad):

\[
R_A = \frac{43.43\times10^{-3}}{2\times206265}\,(221.72)(3.086\times10^{16})
   \approx 7.20\times10^{11}~\mathrm{m}
\]

\[
R_B = \frac{1.43\times10^{-3}}{2\times206265}\,(2.544)(3.086\times10^{16})
   \approx 2.72\times10^{8}~\mathrm{m}
\]

Expressed in solar radii ($R_\odot=6.96\times10^{8}$ m):

\[
R_A \approx 1.035\times10^{3}R_\odot, \qquad
R_B \approx 0.391R_\odot
\]

\subsubsection*{(c) Surface Gravity}

\[
g = \frac{GM}{R^2}
\]

\[
M_A = 19.6\,M_\odot = 3.89\times10^{31}\,\mathrm{kg}, \qquad
M_B = 0.46\,M_\odot = 9.15\times10^{29}\,\mathrm{kg}
\]

\[
g_A = \frac{6.67\times10^{-11}(3.89\times10^{31})}{(7.20\times10^{11})^2}
     \approx 5.0\times10^{-3}~\mathrm{m/s^2}
\]

\[
g_B = \frac{6.67\times10^{-11}(9.15\times10^{29})}{(2.72\times10^{8})^2}
     \approx 8.2\times10^{2}~\mathrm{m/s^2}
\]

\subsubsection*{(d) Luminosity Class Comparison}

We expect the main-sequence dwarf (V) to have much higher surface gravity than the supergiant (Ia).  
Indeed, $g_B \gg g_A$, consistent with their classifications.

\subsubsection*{(e) Temperatures}

Using
\begin{equation}
B - V = m_B - m_V
\tag{13.35}
\end{equation}
and
\begin{equation}
T \approx \frac{9000~\mathrm{K}}{(B - V) + 0.93}
\tag{13.36}
\end{equation}

\[
T_A = \frac{9000}{(2.27 - 0.42) + 0.93} \approx 3.24\times10^{3}~\mathrm{K}, \qquad
T_B = \frac{9000}{(8.96 - 7.52) + 0.93} \approx 3.80\times10^{3}~\mathrm{K}
\]

\subsubsection*{(f) Luminosities}

From the Stefan–Boltzmann law:
\[
L = 4\pi R^2\sigma_\mathrm{SB}T^4
\]

\[
L_A = 4\pi(7.20\times10^{11})^2(5.67\times10^{-8})(3.24\times10^{3})^4
     \approx 4.06\times10^{31}~\mathrm{W}
\]

\[
L_B = 4\pi(2.72\times10^{8})^2(5.67\times10^{-8})(3.80\times10^{3})^4
     \approx 1.10\times10^{25}~\mathrm{W}
\]

\subsubsection*{(g) H–R Diagram Location}

Star A: very cool, extremely luminous $\rightarrow$ upper right of the H–R diagram (red supergiant).  
Star B: cooler main-sequence dwarf $\rightarrow$ lower right region (red dwarf).  
Both positions are consistent with their spectral classifications.

\begin{table}[htbp]
\makebox[\linewidth][c]{%
\begin{tabular}{lcccccc}
\hline
\textbf{Name} & \textbf{Distance (pc)} & \textbf{Radius (m)} & \textbf{Radius ($R_\odot$)} &
\textbf{$g$ (m/s$^2$)} & \textbf{Temp (K)} & \textbf{$L_{\text{tot}}$ (W)} \\
\hline
Star A & 221.72 & $7.20\times10^{11}$ & $1.04\times10^{3}$ & $5.0\times10^{-3}$ & $3.24\times10^{3}$ & $4.06\times10^{31}$ \\
Star B & 2.544  & $2.72\times10^{8}$  & $3.91\times10^{-1}$ & $8.2\times10^{2}$ & $3.80\times10^{3}$ & $1.10\times10^{25}$ \\
\hline
\end{tabular}}
\end{table}


\subsection{Stellar main sequence lifetimes}
Given \dots
\begin{equation}
    N_H = \frac{M}{m_p} 
    \tag{15.52}
\end{equation}
\begin{equation}
    E_{fus} = \frac{N_h}{4}\Delta E
    \tag{15.53}
\end{equation}

\begin{equation}
    t_{fus} = \frac{E_{fus}}{L}
    \tag{15.54}
\end{equation}

We can derive \dots
    \[t_{fus} = \frac{1}{L} \frac{1}{4} \frac{\Delta E}{1} \frac{M}{m_p}\]
    \[t_{fus} = \frac{\Delta E \times M}{4L\times m_p}\]

Where \dots
    \[\Delta E = 4.1 \times 10^{-12} \text{J}, \qquad m_p = 1.67 \times 10^{-27} \text{kg}\]
\subsubsection{High-Mass}
Given \dots
\[M = 100 M_\odot \approx 1.989 \times 10^{32} \text{kg}, \qquad  L = 10^6 L\odot \approx 3.83 \times 10^{32} \text{W}\]
\[t_{fus} = \frac{4.1 \times 10^{-12} \text{J} \times 1.989 \times 10^{32} \text{kg}}{4(3.83 \times 10^{32} \text{W})\times 1.67 \times 10^{-27} \text{kg}} \approx 3.187 \times 10^{14} \text{s}\]
\subsubsection{Low-mass}
Given \dots
\[M = .5 M_\odot \approx 9.945 \times 10^{29} \text{kg}, \qquad  L = .1 L\odot \approx 3.83 \times 10^{25} \text{W}\]
\[t_{fus} = \frac{4.1 \times 10^{-12} \text{J} \times 9.945 \times 10^{29} \text{kg}}{4(3.83 \times 10^{25} \text{W})\times 1.67 \times 10^{-27} \text{kg}} \approx 1.58 \times 10^{19} \text{s}\]
\subsection{}

\subsection{Extra Credit}
\subsubsection{}
The parallax error would affect its location on the y-axis. This is because the parallax would affect the distance measured, which in turn effect the absolute luminosity, hence shifting the location on the y-axis.\footnote{It could also affect the x-axis through redshift + Hubble const, but if a parallax error is that big you might have bigger problems.}

\subsubsection{}
Given \dots
\[d = \frac{1}{\theta_{par}} \qquad F = \frac{L_{abs}}{4\pi d^2} \]

We can find the difference as \dots
\[
\frac{L_{\rm meas}}{L_{\rm true}}
= \left(\frac{d_{\rm meas}}{d_{\rm true}}\right)^2
= \left(\frac{1}{0.9}\right)^2
= 1.234\dots
\]

\subsubsection{}
\[\approx 23\%_{error}\]

\begin{center}
    This means that the L value on the y-axis of the HR diagram should be \(23\%\) brighter, meaning the star should be further \textbf{up} on the HR diagram.
\end{center}

% \subsection{}

% \begin{table}[htbp]
% \makebox[\linewidth][c]{%
% \begin{tabular}{lcccccc}
% \hline
% \textbf{Name} & \textbf{Spectral type} & $m_B$ (mag) & $m_V$ (mag)
% & \textbf{Diameter} ($\times 10^{-3}$") & \textbf{Parallax} (") & \textbf{Mass} ($M_\odot$) \\
% \hline
% Star A & M2 Ia & 2.27 & 0.42 & 43.43 & 0.00451 & 19.6 \\
% Star B & M2 V  & 8.96 & 7.52 & 1.43 & 0.393   & 0.46 \\
% \hline
% \end{tabular}}
% \end{table}

% \begin{table}[htbp]
% \makebox[\linewidth][c]{%
% \begin{tabular}{lcccccc}
% \hline
% \textbf{Name} & \textbf{Distance (pc)} & \textbf{Radius (m)} & \textbf{Radius} ($R_\odot$)
% & $\mathbf{g_s}$ (m/s$^2$) & \textbf{Temp (K)} & $\mathbf{L_{\text{tot}}}$ \\
% \hline
% Star A & 221.72 & $1.44\times10^{12}$ & 2069.9 & 0.00125 & 3272.7 & $1.70\times10^{32}$ \\
% Star B & 2.544  & $5.44\times10^{8}$  & 0.782   & 206.2   & 3787.5 & $4.39\times10^{25}$ \\
% \hline
% \end{tabular}}
% \end{table}

% \subsubsection{}
% \[ d(\mathrm{Parsecs}) = \frac{1}{\theta \mathrm{(Parallax(arcsec))}}\]
% \[d_a = \frac{1}{.00451} \approx 221.72 \mathrm{parsecs}, \qquad d_a = \frac{1}{.393} \approx 2.544 \mathrm{parsecs}\] 

% \subsubsection{}
% Given \dots
% \[\theta_{angRes} = \frac{R}{d}\]
% \[ \theta d  = R \]
% With \(\theta_a = 43.43 \times 10^{-3}\)(arcsec) and \(d_a \approx 221.72\)pc \dots
% \[R_a = \frac{43.43 \times 10^{-3}}{3600} \frac{\pi}{180} \times 221.72 * 3.086 \times 10^{16}  \approx 1.44\times 10^{12} \]

% With \(\theta_b =  1.43 \times 10^{-3}\)(arcsec) and \(d_b \approx 2.544\)pc \dots
% \[R_b = \frac{1.43 \times 10^{-3}}{3600} \frac{\pi}{180} \times 2.544 * 3.086 \times 10^{16}  \approx 5.44\times 10^{8} \]

% \[\frac{R}{R_s} = R_\odot\]
% With \(R_s = 6.96\times 10^8\)m
% \[R_a \approx 2069.9 R_\odot\] 
% \[R_b \approx .782 R_\odot\] 

% \subsubsection{}
% Given \dots
% \[g = \frac{GM}{R^2}\]
% With \dots
%     \[M_a = 19.6M_\odot \times 1.989 \times 10^{30}kg \approx 3.89 \times 10^{31} \mathrm{kg}\]
%    \[M_b = .46M_\odot \times 1.989 \times 10^{30}kg \approx  9.1494 \times 10^{29}\mathrm{kg}\]
% and radii from table, we can calculate \dots
% \[g_a =  \frac{6.67 \times 10^{-11} \times 3.89\times10^{31}}{(1.44\times 10^{12})^2} \approx .00125 \mathrm{m/s^2}\]
% \[g_a =  \frac{6.67 \times 10^{-11} \times 9.1494\times10^{29}}{(5.44\times 10^{8})^2} \approx 206.2 \mathrm{m/s^2}\]

% \subsubsection{}
% We would expect star A to have a greater mass. This is because it is a much lower magnitude then star B. 

% \subsubsection{}
% Given \dots
% \begin{equation}
%     B - V = M_b - M_v
%     \tag{13.35}
% \end{equation}
% \begin{equation}
%     T \approx \frac{9000\mathrm{K}}{(B - V) + 0.93}
%     \tag{13.36}
% \end{equation}
% We can calculate \(T_a\) and \(T_b\) as \dots
% \[T_a \approx \frac{9000\mathrm{K}}{(2.27 -.42) + .93} \approx 3272.7\text{K}, \qquad T_b \approx \frac{9000\mathrm{K}}{(8.96 -7.52) + .93} \approx 3787.5\text{K}\]

% \subsubsection{}
% Given the Stefan-Boltzmann Law \dots

% \[L=4\pi R^2 \sigma_{sb} T^4\]

% \[L_a = 4\pi (1.44 \times 10^{12})^2 \sigma_{sb} (3272.7)^4 \approx 1.700 \times 10^{32} \text{W}\]

% \[L_b = 4\pi (5.44 \times 10^{8})^2 \sigma_{sb} (3787.5)^4 \approx 4.39 \times 10^{25} \text{W}\]
% \subsubsection{}




\end{document}